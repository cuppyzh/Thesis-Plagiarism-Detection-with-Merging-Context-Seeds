\documentclass[../Proposal.tex]{subfiles}
 
\begin{document}
	\section{Pengujian}
	Pengujian sistem yang dibangun dilakukan menggunakan \textit{Tira\cite{tira}} yang mengacu pada paper acuan\cite{mcs-paper,acuan-3}. Dimana nilai perfomansi dihitung berdasarkan akurasi pengklasifikasian oleh sistem. Kelas klasifikasi yang ada adalah :
	
	\begin{enumerate}
		\item \textit{\textbf{No-Plagiarism}}
		\item \textit{\textbf{No-Obfuscation}}
		\item \textit{\textbf{Random-Obfuscation}} 
		\item \textit{\textbf{Summary-Obfuscation}} 
	\end{enumerate}
	
	Dimana karakteristik dari kelas yang ada sudah dijelaskan pada Bab 2. Nilai akurasi didapatkan dengan menggunakan  \textit{Precision, Recall, F-Measure}. Dan pada akhir sistem, data yang akan di\textit{outputkan} ditunjukan oleh tabel \ref{tabel-perfomansi}.
	
	\begin{table}[h!]
		\centering
		\label{tabel-perfomansi}
		\caption{Tabel Perfomansi Sistem}
		\begin{tabular}{ | c | c | c | c | c | c | }
			\hline
			\textbf{Corpus} & \textbf{Pairs} & \textbf{PlagDet} & \textbf{Precision} & \textbf{Recall} & \textbf{Granularity} \\  
			\hline
			pan14-training-corpus &  &  & &  & 
			\\ \hline 
		\end{tabular}
		
		
	\end{table}
\end{document}