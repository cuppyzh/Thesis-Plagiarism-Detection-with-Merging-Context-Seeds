\documentclass[../Proposal.tex]{subfiles}
 
\begin{document}
	\section{Rencana Kegiatan}
	\noindent Adapun metedeologi yang digunakan untuk memecahkan masalah yang ada yaitu sebagai berikut :
	\begin{enumerate}
		\item Studi Literatur\\
		Pada tahap ini dikumpulkan data dan informasi segala metode yang dibutuhkan untuk menyelesaikan masalah yang ada. Studi literatur yang digunakan diantaranya adalah sebagai berikut : 
		\begin{enumerate}
			\item \textit{Plagiarism}.
			\item \textit{Text Alignment}.
			\item \textit{Merging Context Seed}.
			\item \textit{N-Gram}.
			\item \textit{K-Nearest Neighbor}.
		\end{enumerate}
		
		\item Analisis Kebutuhan Sistem\\
		Dilakukan analisis terhadap kebutuhan sistem untuk mencapai tujuan pada tugas akhir ini.
		
		\item Perancangan Sistem\\
		Merancang alur sistem yang akan dibangun untuk tugas akhir ini, dimulai dari \textit{input} berupa \textit{pairs} dan dokumen terindikasi berikut sumbernya kemudian implementasi metode hingga mengeluarkan \textit{output}.
		
		\item Implementasi\\
		Mengimplementasikan metode yang dipelajari kedalam sistem, disini segala proses mulai dari \textit{reprocessing}, ekstraksi ciri menggunakan \textit{N-Gram}, perhitungan menggunakan \textit{Merging Context Seed} dan klasifikasi menggunakan \textit{K-Nearest Neighbor}.
		
		\item Pengujian Sistem\\
		Melakukan uji coba dengan menjalankan sistem yang telah dibuat dan melakukan analisis sementara menggunakan \textit{f-measure}.
		
		\item Analisis\\
		Menganalisis hasil \textit{output} yang dikeluarkan dari sistem. Menghitung akurasi dengan \textit{f-measure}.
		
		\item Pembuatan Laporan\\
		Pembuatan laporan mengenai kegiatan dan sistem yang di bangun yang meliputi latar belakang, rumusan masalah, tujuan, implementasi sistem hingga hasil analisis yang dilakukan selama pengerjaan tugas akhir.
	\end{enumerate}
\end{document}