\documentclass[../Proposal.tex]{subfiles}
 
\begin{document}
	\section{Latar Belakang}
	\textit{Plagiarism}, Plagiat atau Penjiplakan merupakan tindakan berupa mengkalim suatu ide, pendapat maupun karangan orang lain sebagai hasil karya sendiri \cite{plagiat-kbbi}. Tindak plagiat ini merupakan masalah yang sangat serius, dan sering ditemukan pada bidang sastra dan pendidikan. \\ 
	
	\noindent Tindak plagiat ini juga sangat merugikan baik pihak penjiplak dan pihak yang dijiplak. Merugikan pihak penjiplak karena dapat berujung pada sanksi pidana dan secara tidak langsung membatasi kreatifitas si penjiplak itu sendiri. Dan merugikan pihak yang dijiplak karena usaha atau hasil karya yang ia hasilkan digunakan orang dengan seenaknya tanpa memberikan \textit{credits} kepada pengarang aslinya. \\ 
	
	Tindakan plagiat ini juga terdapat beberapa karakteristik, yaitu : 
	
	\begin{enumerate}
		\item Mengklaim ide / gagasan orang lain miliknya sendiri.
		\item Menggunakan tulisan orang lain pada sebagian / seluruh karyanya secara utuh.
		\item Menggunakan tulisan orang lain pada sebagian / seluruh karyanya dengan menambahkan atau menghilangkan beberapa kata pada tulisannya.
		\item Menggunakan tulisan orang lain pada sebagian / seluruh karyanya dengan mengubah tataan kalimatnya namun mempunyai makna yang sama.
	\end{enumerate}
	
	\noindent Dengan karakteristik diatas, suatu tindak plagiat dapat diketahui dengan menggunakan pendekatan \textit{Text Alignment} pada kasus \textit{Plagiarism Detection} atau Deteksi Plagiat sesuai dengan \textit{task} yang ada di PAN\cite{pan-task-2014}. Hal ini dapat dilakukan karena dengan pendekatan \textit{Text Alignment} dapat diketahui apakah suatu dokumen menjiplak dokumen lain berdasarkan \textit{pairs} dari proses \textit{Source Retrieval}, hal ini dengan membaca pola kalimat beserta kata-kata yang ada pada dua buah dokumen. Dan dari pendekatan \textit{Text Alignment} terdapat metode yang bernama \textit{Merging Context Seed} yang akan dibahas pada Tugas Akhir ini. \\
	
	\noindent Diharapkan dengan metode yang diusulkan ini dapat dibangun sistem yang dapat mendeteksi tindak plagiat dengan akurat berdasarkan info \textit{pairs} dari proses \textit{Source Retrieval}. Pada sistem yang dibangun juga akan digunakan \textit{N-Gram} sebagai metode bantuan untuk mengekstraksi ciri dari dua dokumen yang ada.
\end{document}