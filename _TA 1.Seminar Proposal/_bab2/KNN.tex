\documentclass[../Proposal.tex]{subfiles}
 
\begin{document}
	\section{\textit{K-Nearest Neighbor}}
	\textit{K-Nearest Neighbor} merupakan algoritma yang biasanya digunakan untuk mengklasifikasi suatu data berdasarkan pola yang ada. Algoritma ini mengklasifikasikan data berdasarkan label yang paling mirip dengan data yang ada di \textit{training set}. Pada umumnya algoritma ini menggunakan \textit{Euclidean Distance} untuk mengukur tingkat kesamaan data\cite{knn}.\\ 
	Algoritma ini dapat diadaptasikan ke berbagai masalah yang ada, sehingga \textit{classifier} ini dipilih untuk pengerjaan tugas akhir ini.
\end{document}