\documentclass[../Proposal.tex]{subfiles}
 
\begin{document}
	\section{\textit{Plagiarism Detection}}
	\textit{Plagiarism Detection} merupakan solusi yang ditawarkan untuk menangani kasus \textit{plagiarism}. \textit{Plagiarism Detection} ini dapat mengidentifikasi tindak plagiat dengan beberapa pendekatan. Salah satunya adalah \textit{Text Alignment}. \textit{Plagiarism Detection} ini terbagi menjadi 2 \textit{task} yaitu \textit{\textbf{Source Retrieval}} dan \textit{\textbf{Text Alignment}}\cite{pan-task-2014}.
	
		\subsection{\textit{Source Retrieval}}
		\textit{Source Retrieval} Merupakan \textit{task} awal untuk \textit{Plagiarism Detection}. Pada tahap ini suatu dokumen akan diuji dengan cara melakukan pencarian perkalimat dari dokumen yang diuji. Kaliamt yang diuji akan dimasukan kedalam \textit{query} mesin pencarian yang berisikan jurnal atau dokumen sejenis. Kemudian apabila ada kemiripan akan dibuat file yang berisikan informasi pasangan dokumen terindikasi dengan dokumen sumber\cite{information-retrieval}.
		
		\subsection{\textit{Text Alignment}}
		\textit{Text Alignment} merupakan pendekatan yang dapat menguji kebenaran hasil pemasangan dari tahap sebelumnya. Pada tahap ini hasil dari proses \textit{Source Retrieval} akan diuji kebenarannya, apakah dokumen tersebut terbukti memplagiat dokumen sumber atau tidak dengan cara mengekstrasi ciri yang ada pada dokumen yang terindikasi dan dokumen sumber yang kemudian diolah dengan metode tertentu. 
\end{document}