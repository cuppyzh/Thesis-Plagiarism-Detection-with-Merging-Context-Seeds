\documentclass[../Proposal.tex]{subfiles}
 
\begin{document}
	\section{\textit{Plagiarism}}
	\textit{Plagiarism} merupakan tindakan mengklaim suatu ide, gagasan ataupun tulisan orang lain sebagai miliknya sendiri. Gagasan atau tulisan yang di klaim dapat berupa jurnal, buku, ucapan ataupun hasil diskusi. \\
	
	\noindent Tindak \textit{plagiarism} ini dapat berupa menghilangkan atau menambahkan satu atau beberapa kata dari tulisan asli seorang penulis dan digunakan untuk karya / tulisan sendiri\cite{rule-plagiat}. Contohnya adalah : \\
	
	\indent \textbf{Teks asli : } \textit{Dengan menggunakan metode \textbf{\textit{k-Nearest Neighbor}} kita dapat mengetahui derajat ketetangaan suatu \textit{node} dengan \textit{node} lainnya.}
	
	\indent \textbf{Teks Plagiat : } \textit{Dengan metode \textbf{\textit{k-Nearest Neighbor}} kita dapat mengetahui derajat ketetangaan suatu \textit{node} dengan \textit{node} lain disekitarnya. \\}
	
	Selain itu tindak \textit{plagiarism} dapat berupa \textit{paraphrase}. Yaitu mengubah tataan suatu kalimat menjadi bentuk lain, namun masih memiliki makna yang sama. Contohnya adalah : \\
	
	\indent \textbf{Teks asli : } \textit{Dengan menggunakan metode \textbf{\textit{k-Nearest Neighbor}} kita dapat mengetahui derajat ketetangaan suatu \textit{node} dengan \textit{node} lainnya.}
	
	\indent \textbf{Teks Plagiat : } \textit{Untuk menghitung derajat ketetanggaan suatu \textit{node} dapat menggunakan metode \textbf{\textit{k-Nearest Neighbor}}.} \\
	
	\noindent Tetapi walaupun suatu teks diubah dengan \textit{paraphrase} tindak \textit{plagiarism} masih dapat di indetifikasi karena ada kemiripan makna antara dua buah kalimat.
\end{document}