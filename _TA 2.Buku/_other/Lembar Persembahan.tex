\documentclass[../Book.tex]{subfiles}
 
\begin{document}
	\section*{\centerline{Lembar Persembahan}}
	
	Yang Utama Dari Segalanya,\\
	\noindent Sembah sujud serta syukur kepada \textbf{Allah SWT} yang membekali penulis dengan ilmu yang berlimpah, dan karna izin-Nya pula lah penelitian ini dapat selesai. Sholawat serta salam selalu terlimpahkan keharibaan \textbf{Rasullah Muhammad SAW}. \\
	
	\noindent \textbf{Pembimbing Tugas Akhir} \\
	Pak Moch. Arif Bijaksana dan Pak Syahrul Mubarok selaku pembimbing dalam pengerjaan tugas akhir ini. Terimakasih pak untuk bimbingan dan pencerahannya. Tanpa bapak mungkin tugas akhir ini tidak akan selesai. Mohon maaf bila ada kekurangan selama penulis menjadi murid bimbingan bapak. \\
	
	\noindent \textbf{Dosen Wali dan IF-36-02} \\
	Terima kasih untuk Bu Tisa (Siti Saadah) yang sudah membantu dan memberikan arahan selama masa kuliah. Dan anak-anak IF-36-02 yang menjadi rumah selama masa kuliah ini. Terutama Rizki yang sering berbagi ilmu baik didalam maupun luar kuliah.\\
	
	\noindent \textbf{Kontrakan Bahagia} \\
	Babeh, Ali, Lutpi dan Woempa yang menemani hidup dikosan, dan bertahan hidup di dakol. Semoga semuanya cepet lulus yaaa. \\
	
	\noindent \textbf{Keluarga Besar BASDAT dan IFLAB} \\ 
	Terima kasih sudah diberi tempat untuk mengasah kemampuan. Terima kasih terutama untuk \textit{Foya Foya, Kamar Kos, ASLAB 15/16 dan ASLAB 16/17}. Semoga semua cepet lulus juga yaaa. \\
	
	\noindent Dan kepada seluruh pihak yang tidak sempat penulis sebutkan, semoga kesuksesan selalu menyertai kita semua. \\
	
	\noindent Dan yang tidak mungkin terlupa, \\
	Terima kasih untuk \textbf{Papah}, \textbf{Mamah}, \textbf{Teteh}, \textbf{Bebi}, dan \textbf{Gege}. Yang selalu mendukung dan memberi semangat tidak hanya selama pengerjaan tugas akhir ini tetapi juga selama menjalani kuliah dan kehidupan ini. Dan terima kasih juga udah nanyain tiap hari gimana tugas akhirnya, yang akhirnya selesai. Terima Kasih. \textit{This is for you.}
	
\end{document}