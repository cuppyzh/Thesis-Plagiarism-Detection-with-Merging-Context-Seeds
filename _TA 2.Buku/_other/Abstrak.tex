\documentclass[../Book, Implementasi Algoritma Merging Context Seeds untuk Plagiarism Detection.tex]{subfiles}
 
\begin{document}
	\section*{\centerline{Abstrak}}
	
	Plagiat merupakan masalah yang sering ditemukan di masyarakat, bahkan menurut survey 89\% responden sering menemukan kasus plagiat pada bidangnya masing-masing. Tindak plagiat ini dapat berupa mengambil tulisan orang lain yang digunakan untuk kepentingan diri sendiri. Adapun salah satu pendekatan yang dapat dilakukan untuk mendeteksi tindak plagiat ini adalah dengan \textit{Text Alignment}. Sehingga pada penelitian ini diusung salah satu metode yaitu \textit{Merging Context Seeds} yang bekerja dengan cara menggabungkan ciri yang ada pada \textit{suspicious-document} dan \textit{source-document} dengan metode ekstraksi ciri \textit{n-skip-k-grams}. Dengan diimplementasikannya metode \textit{Merging Context Seeds}, penilitian ini mendapat nilai \textit{$F_{1}$} sebesar 0.532. \\
	
	\noindent \katakunci{\textit{merging context seeds}, \textit{seeds}, \textit{merge}}
\end{document}