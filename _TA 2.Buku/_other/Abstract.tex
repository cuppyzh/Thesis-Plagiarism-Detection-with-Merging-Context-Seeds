\documentclass[../Book, Implementasi Algoritma Merging Context Seeds untuk Plagiarism Detection.tex]{subfiles}
 
\begin{document}
	\section*{\centerline{\textit{Abstract}}}
	
	\textit{Plagiarism is a case that can be found in society especially in educational field, even on survey 89\% responden often found plagiarism case in their own field. Plagiarsm can be form as using other people writing, or idea as their own to gain benefit from it. As for, there is an approach that can be done to detect plagiarism which is Text Alignment. Therefore in this research, researcher submit method called Merging Context Seeds that works by merge feature between suspicous-document and source-document generated by n-skip-k-grams feature extraction method. With the implementation of Merging Context Seeds method, this research get $F_{1}$-Score 0.532.} \\
	
	\noindent \keywords{\textit{merging context seeds, seeds, merge}}
\end{document}