\documentclass[../Book.tex]{subfiles}
 
\begin{document}
\section{Pengujian Sistem}
Pengujian sistem dilakukan dengan 4 skenario. Skenario ini didapat dari jenis plagiat yang dijelaskan pada Bab 1 dan Bab 2. 

\begin{enumerate}
	\item Skenario 1 : 500 \textit{pair} berisi pasangan dokumen \textit{\textbf{No Plagiarism}}.
	\item Skenario 2 : 500 \textit{pair} berisi pasangan dokumen \textit{\textbf{No Obfuscation}}.
	\item Skenario 3 : 500 \textit{pair} berisi pasangan dokumen \textit{\textbf{Random Obfuscation}}.
	\item Skenario 4 : 500 \textit{pair} berisi pasangan dokumen \textit{\textbf{Translation Obfuscation}}.
\end{enumerate}

Pada akhir pengujian setiap skenario akan dianalisis dan dihitung perfomansi dari sistem ke dalam tabel berikut :

% Please add the following required packages to your document preamble:
% \usepackage{booktabs}
\begin{table}[H]
	\centering
	\caption{My caption}
	\label{my-label}
	\begin{tabular}{@{}lllll@{}}
		\toprule
		Tipe Plagiat            & Jumlah Data & \textit{Precision} & \textit{Recall} & \textit{F1} \\ \midrule
		No Plagiarism           &             &                    &                 &             \\
		No Obfuscation          &             &                    &                 &             \\
		Random Obfuscation      &             &                    &                 &             \\
		Translation Obfuscation &             &                    &                 &             \\ \bottomrule
	\end{tabular}
\end{table}

\end{document}