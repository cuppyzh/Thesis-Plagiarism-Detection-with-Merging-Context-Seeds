\documentclass[../Book.tex]{subfiles}
 
\begin{document}
\section{Analisis Kebutuhan}
Analisis kebutuhan mencakup dari penentuan spesifikasi perangkat yang akan digunakan, baik perangkat lunak maupun perangkat keras.

\subsection{Kebutuhan Fungsional}
Kebutuhan fungsional sistem diantaranya adalah : 
\begin{enumerate}
	\item Sistem dapat membaca \textit{input plain-text} yaitu \textit{suspicious-document}, \textit{source-document}, dan \textit{pairs}.
	\item Sistem mengolah data yang masuk ke dalam sistem dengan metode \textit{Skipwrod-grams}, \textit{Merging Context Seeds} dan \textit{Single-linkage Clustering}.
	\item Menampilkan \textit{log}, dan hasil akhir berupa \textit{highlight} pada \textit{suspicious-document} dan \textit{source-document} apabila ditemukan tindak plagiat.
\end{enumerate}

\subsection{Kebutuhan Non-Fungsional}

\subsubsection{Spesifikasi Perangkat Lunak}
Perangkat lunak yang akan digunakan untuk tugas akhir ini adalah sebagai berikut : 
\begin{enumerate}
	\item Sitem Operasi : Windows 10 64-bit.
	\item Bahasa Pemrograman : Python 2.7 \\
	Daftar library yang diperlukan untuk menjalankan sistem yang dibangun ditunjukan pada Tabel \ref{tab:lib}.
	
	% Please add the following required packages to your document preamble:
	% \usepackage{booktabs}
	\begin{table}[H]
		\centering
		\caption{Daftar \textit{Library} yang Digunakan}
		\label{tab:lib}
		\begin{tabular}{@{}ll@{}}
			\toprule
			\textit{Library}               & Fungsi                                 \\ \midrule
			nltk                  & Daftar \textit{stopwords} dalam bahasa Inggris  \\
			xml.etree.ElementTree & Membaca informasi xml dari \textit{datasets}         \\
			pyexcel\_xls          & Menyimpan hasil pengujian ke dalam file Excel \\ \bottomrule
		\end{tabular}
	\end{table}
	
	\item \textit{Tools :} Sublime Text 3, Git Bash.
	
\end{enumerate}
\end{document}