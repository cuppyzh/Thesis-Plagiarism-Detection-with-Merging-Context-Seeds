\documentclass[../Book, Implementasi Algoritma Merging Context Seeds untuk Plagiarism Detection.tex]{subfiles}
 
\begin{document}
	\section{Latar Belakang}
	Menurut hasil survey yang dilakukan oleh iThenticate\cite{it}, sebanyak $89\%$ responden yang ditemui menjawab sering menemui kasus plagiat pada bidangnya masing-masing. Dan lebih dari $25\%$ responden menyatakan bahwa plagiat merupakan masalah serius yang harus diselesaikan. Namun dengan makin banyaknya dokumen yang terkumpul akan semakin sulit mendeteksi tindak plagiat secara manual. \textit{Text alignment} adalah solusi yang ada untuk menyelesaikan masalah plagiat yang ada, dengan cara membangkitkan bagian pada dua buah dokumen yang terindikasi plagiat. \\
	
	\noindent Paragraf atau kalimat yang serupa yang digunakan pada dua buah dokumen dapat dideteksi dari penggunaan kata dan penataannya. Hal ini yang menjadi alasan pendekatan \textit{text alignment} dapat dilakukan, yaitu dengan menjajarkan seluruh fitur yang ada pada dua buah dokumen dan mencari irisan antara dua buah dokumen tersebut. Dari seluruh metode \textit{text alignment} yang ada, dipilih \textit{Merging Context Seeds} yang merupakan salah satu metode yang diajukan pada PAN\cite{pan-task-2014} yang memiliki nilai perfomansi \textit{plagdet} 0.826. \textit{Merging context seeds} terfokus kepada \textit{seeds} yang merupakan fitur yang beririsan antara dua buah dokumen, dan melakukan \textit{clustering} atau pengelompokan data pada seeds yang ada. Dari kelompok data yang didapat, dipilih yang merupakan tindak plagiat. \\
	
	\noindent Permasalahan yang akan diselesaikan pada penilitian ini adalah, bagaimana mengimplementasikan algoritma \textit{merging context seeds} dan membangun sistem yang mampu mendeteksi tindak plagiat dari berbagai tipe tindak plagiat yang ada secara akurat.

\end{document}