\documentclass[../Book, Implementasi Algoritma Merging Context Seeds untuk Plagiarism Detection.tex]{subfiles}
 
\begin{document}
	\section{Metodologi}
	\noindent Adapun metodologi yang digunakan untuk memecahkan masalah yang ada yaitu sebagai berikut :
	\begin{enumerate}
		\item Studi Literatur\\
		Mengumpulkan dan mempelajari kajian yang digunakan untuk menyelesaikan masalah yang ada, yang dapat membantu dalam metode \textit{Merging Context Seeds}.
		
		\item Analisis Kebutuhan Sistem\\
		Dilakukan analisis terhadap kebutuhan sistem untuk mencapai tujuan pada tugas akhir ini.
		
		\item Perancangan Sistem\\
		Merancang alur sistem yang akan dibangun untuk tugas akhir ini, dimulai dari \textit{input}, proses, hingga \textit{output}.
		
		\item Implementasi\\
		Mengimplementasikan metode yang dipelajari kedalam sistem, mulai dari \textit{preprocessing}, ekstraksi ciri, implementasi \textit{Merging Context Seed} dan \textit{clustering} menggunakan \textit{Agglomerative single-link clustering}.
		
		\item Pengujian Sistem\\
		Melakukan uji coba dengan menjalankan sistem yang telah dibuat dan melakukan analisis sementara.
		
		\item Analisis\\
		Menganalisis hasil \textit{output} yang dikeluarkan dari sistem. Menghitung nilai \textit{precision} dan \textit{recall} berdasarkan hasil \textit{running} dari \textit{datasets} yang ada.
		
		\item Penyusunan Laporan\\
		Pembuatan laporan mengenai kegiatan dan sistem yang di bangun yang meliputi latar belakang, rumusan masalah, tujuan, implementasi sistem hingga hasil analisis yang dilakukan selama pengerjaan tugas akhir.
	\end{enumerate}
\end{document}