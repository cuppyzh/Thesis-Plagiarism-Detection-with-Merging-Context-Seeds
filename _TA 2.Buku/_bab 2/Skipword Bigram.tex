\documentclass[../Proposal.tex]{subfiles}
 
\begin{document}
	\section{\textit{Skip Gram}}
	\textit{Skip Gram} atau yang sering disebut dengan k-skip-n-grams merupakan salah satu metode untuk mengekstraksi ciri dari suatu dokumen yang juga merupakan bentuk lain dari \textit{N-Gram}. Dengan \textit{skip gram} rangkaian karakter/kata yang ada pada dokumen akan di lompati untuk mendapatkan fitur. \textit{k} menunjukan jumlah karakter/kata yang dilompati. Sedangkan \textit{n} menunjukan panjang karakter untuk satu buah fitur. Fitur hasil dari metode k-skip-n-grams sendiri dapat didefinisikan pada Persamaan \ref{eq:k-skip-n-grams}\cite{skipgram}.

	\begin{center}
	\begin{equation}
		\{ w_{i_{1}}, w_{i_{2}}, ... w_{i_{n}} | \sum_{j=1}^{n} i_{j}-i_{j=1} < k \}
		\label{eq:k-skip-n-grams}
	\end{equation} 
	\end{center}

	\noindent 

	\noindent Sebagai contoh, kalimat : 

	\begin{center}
		\textit{" A sentence is a group of words that are put together to mean something. "}
	\end{center}

	\noindent Apabila kalimat diatas di ekstraksi menggunakan 2-skip-2-grams maka akan didapat fitur sesuai pada Tabel \ref{generated-feature} :


		\begin{table}[H]
			\small
			\caption{Contoh Ekstraksi Ciri dengan 2-skip-2-grams}
			\label{generated-feature}
			\centering
			\begin{tabular}{cl}
				\hline
				\multicolumn{2}{c}{Daftar Fitur}   \\ \hline
				\multicolumn{2}{c}{A\_IS}          \\
				\multicolumn{2}{c}{SENTENCE\_A}    \\
				\multicolumn{2}{c}{IS\_GROUP}      \\
				\multicolumn{2}{c}{A\_OF}          \\
				\multicolumn{2}{c}{GROUP\_WORDS}   \\
				\multicolumn{2}{c}{OF\_THAT}       \\
				\multicolumn{2}{c}{WORDS\_ARE}     \\
				\multicolumn{2}{c}{THAT\_PUT}      \\
				\multicolumn{2}{c}{ARE\_TOGETHER}  \\
				\multicolumn{2}{c}{PUT\_TO}        \\
				\multicolumn{2}{c}{TOGETHER\_MEAN} \\
				\multicolumn{2}{c}{TO\_SOMETHING}  \\
				\multicolumn{2}{c}{MEAN\_*}        \\
				\multicolumn{2}{c}{SOMETHING\_*}   \\ \hline
			\end{tabular}

		\end{table}
	
\end{document}