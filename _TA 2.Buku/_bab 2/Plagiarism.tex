\documentclass[../Proposal.tex]{subfiles}
 
\begin{document}
	\section{\textit{Plagiarism}}
	\textit{Plagiarism} atau plagiat merupakan tindakan mengklaim suatu ide, gagasan ataupun tulisan orang lain sebagai miliknya sendiri. Gagasan atau tulisan yang di klaim dapat berupa jurnal, buku, ucapan ataupun hasil diskusi. \\
	
	\noindent Tindak plagiat pada hasil tulisan dapat berupa menghilangkan atau menambahkan satu, atau beberapa kata dari tulisan sumber seseorang dan digunakan pada karya / tulisan orang lain\cite{panduan-anti-plagiarisme}. Kedua teks di bawah ini merupakan contoh tindak plagiat dengan menambah dan mengurangi kata pada teks sumber, \\
	
	\indent \textbf{Teks Asli : } \textit{Dengan menggunakan metode \textbf{\textit{k-Nearest Neighbor}} kita dapat mengetahui derajat ketetangaan suatu \textit{node} dengan \textit{node} lainnya.}
	
	\indent \textbf{Teks Plagiat : } \textit{Dengan metode \textbf{\textit{k-Nearest Neighbor}} kita dapat mengetahui derajat ketetangaan suatu \textit{node} dengan \textit{node} lain disekitarnya. \\}
	
	\noindent Selain itu ada juga \textit{paraphrase}. Yaitu, mengubah tataan suatu kalimat menjadi bentuk lain. Tindak plagiat ini sulit di deteksi karena perubahaan dapat banyak terjadi khususnya apabila ada perubahan kalimat aktif menjadi pasif, atau kebalikannya. Teks di bawah merupakan contoh tindak plagiat \textit{paraphrase}, \\
	
	\indent \textbf{Teks Asli : } \textit{Dengan menggunakan metode \textbf{\textit{k-Nearest Neighbor}} kita dapat mengetahui derajat ketetangaan suatu \textit{node} dengan \textit{node} lainnya.}
	
	\indent \textbf{Teks Plagiat : } \textit{Untuk menghitung derajat ketetanggaan suatu \textit{node} dapat menggunakan metode \textbf{\textit{k-Nearest Neighbor}}.} \\
	
	\noindent Tetapi walaupun suatu teks diubah dengan \textit{paraphrase} tindak \textit{plagiarism} masih dapat di indetifikasi karena ada kemiripan penggunaan kata yang ada pada dokumen \textit{source} dan \textit{suspicious}.
\end{document}